\documentclass[a4paper]{article}

% import packages
\usepackage{multicol} % multiple column
\usepackage{booktabs} % table quality improvement
\usepackage{tabularx} % tabular package
\usepackage{mdframed} % for title
\usepackage{libertine} % libertine font
\usepackage[T1]{fontenc} % font encodings
\usepackage[table]{xcolor} % color manager
\usepackage{tikz} % shaded block
\usepackage[margin=1cm, landscape]{geometry} % document layout

% define new commands for shortcuts
\newcommand{\altp}{Alt$+$}
\newcommand{\ctrlp}{Ctrl$+$}
\newcommand{\shiftp}{Shift$+$}
\newcommand{\cmd}[1]{\$ \textbf{#1}}
\newcommand{\code}[1]{\textbf{#1}}
\newcommand{\hlx}{\\ \midrule[0.3ex]}
\newcommand{\mytoprule}{\toprule[0.5ex]}
\newcommand{\mybottomrule}{\bottomrule[0.5ex]}
\newcommand{\vb}[1]{\tikz[baseline]{\node[rectangle, rounded corners=0.5mm, text=black, fill=black!20, inner sep=1pt, anchor=base]{\$#1}}} % variable box

% document settings
\pagestyle{empty}
\setlength{\parindent}{0pt}
\setlength{\arrayrulewidth}{1pt}
\newcommand{\spacebtwtables}{\newline \vspace*{1em} \newline}

% title settings
\mdfdefinestyle{TitleStyle}{linecolor=gray, linewidth=1pt, leftmargin=0mm, rightmargin=0mm, skipbelow=0mm, skipabove=0mm}

% table settings
\renewcommand\arraystretch{1.2}
\renewcommand{\aboverulesep}{0pt}
\renewcommand{\belowrulesep}{0pt}
\newcommand{\headbf}[1]{\Large\textbf{#1}}
\newcolumntype{T}{>{\columncolor{white}}m{2em}}
\newcolumntype{L}{>{\columncolor{white}[0pt][\tabcolsep]}X}
\newcolumntype{R}{>{\columncolor{white}[\tabcolsep][0pt]}m{19em}}
\newcolumntype{S}{>{\columncolor{white}[\tabcolsep][0pt]}m{16em}}

\begin{document}
\begin{multicols*}{3}

% ---------- Title ---------- %
\begin{mdframed}[style=TitleStyle]
\begin{center}
	\fontfamily{pnc} \selectfont
	\large\scshape
	ETH Robotics Summer School \\
	Linux \& ROS Cheat Sheet \\
	\bigskip
	\footnotesize Author: Yize Wang \\
	\footnotesize Last Updated: \today
\end{center}
\end{mdframed}

\scriptsize
\bigskip

% ---------- File Commands ---------- %
\begin{tabularx}{\linewidth}[ht]{@{}LR@{}}
	\multicolumn{2}{@{}l@{}}{\headbf{File Commands}} \\
	\mytoprule
	\cmd{ls}						& list contents of files and directories \hlx
	\cmd{ls -a}						& list hidden files and directories \hlx
	\cmd{cd \vb{DIR}}				& change working directory to \vb{DIR} \hlx
	\cmd{cd}						& change working directory to home \hlx
	\cmd{mkdir \vb{DIR}}			& create a directory named \vb{DIR} \hlx
	\cmd{pwd}						& print working directory \hlx
	\cmd{rm \vb{FILE}}				& remove \vb{FILE} \hlx
	\cmd{rm -r \vb{DIR}}			& remove \vb{DIR} \hlx
	\cmd{rm -f \vb{FILE}}			& force remove \vb{FILE} \hlx
	\cmd{rm -rf \vb{DIR}}			& force remove \vb{DIR} \hlx
	\cmd{cp \vb{FILE1} \vb{FILE2}}	& copy \vb{FILE1} to \vb{FILE2} \hlx
	\cmd{cp -r \vb{DIR1} \vb{DIR2}}	& copy \vb{DIR1} to \vb{DIR2} \hlx
	\cmd{mv \vb{FILE1} \vb{FILE2}}	& move \vb{FILE1} to \vb{FILE2} \hlx
	\cmd{ln -s \vb{FILE} \vb{LINK}}	& create symbolic link \vb{LINK} to \vb{FILE} \hlx
	\cmd{touch \vb{FILE}}			& create \vb{FILE} \hlx
	\cmd{cat \vb{FILE}}				& view content of \vb{FILE} \hlx
	\cmd{cat > \vb{FILE}}			& write input into \vb{FILE} \hlx
	\cmd{more \vb{FILE}}			& print contents of \vb{FIL}E \hlx
	\cmd{head \vb{FILE}}			& print the first 10 lines of \vb{FILE} \hlx
	\cmd{tail \vb{FILE}}			& print the last 10 lines of \vb{FILE} \hlx
	\cmd{gedif \vb{FILE}}			& edit \vb{FILE} \\
	\mybottomrule
\end{tabularx}
\spacebtwtables

% ---------- System Info ---------- %
\begin{tabularx}{\linewidth}[ht]{@{}LR@{}}
	\multicolumn{2}{@{}l@{}}{\headbf{System Info}} \\
	\mytoprule
	\cmd{env}					& print environment variables \hlx
	\cmd{date}					& print system date and time \hlx
	\cmd{cal}					& print current month calendar \hlx
	\cmd{uptime}				& print system uptime \hlx
	\cmd{w}						& print online users \hlx
	\cmd{whoami}				& print current user \hlx
	\cmd{finger \vb{USER}}		& print information about \vb{USER} \hlx
	\cmd{uname -a}				& print kernel information \hlx
	\cmd{cat /proc/cpuinfo}		& print cpu information \hlx
	\cmd{cat /proc/meminfo}		& print memory information \hlx
	\cmd{man \vb{COMMAND}}		& print user manual of \vb{COMMAND} \hlx
	\cmd{df}					& print disk usage \hlx
	\cmd{du}					& print directory space usage \hlx
	\cmd{free}					& print memory and swap usage \hlx
	\cmd{whereis \vb{APP}}		& print locations of \vb{APP} \hlx
	\cmd{which \vb{APP}}		& print print executable file of \vb{APP} \\
	\mybottomrule
\end{tabularx}
\spacebtwtables

% ---------- Linux Basics ---------- %
\begin{tabularx}{\linewidth}[ht]{@{}LTR@{}}
	\multicolumn{3}{@{}l@{}}{\headbf{Linux Basics}} \\
	\mytoprule
	path symbolic links			&\code{.}					& current directory \\
								&\code{..}					& parent directory \\
								&\code{\textasciitilde}		& home directory \\
								&\code{\textbackslash}		& root directory \hlx
	output direction			&\code{>}					& to a file (rewrite) \\
								&\code{{>}{>}}				& to a file (append) \\
								&\code{|}					& pipe the output of first command to the second \\
	\mybottomrule
\end{tabularx}
\spacebtwtables

% ---------- Compression ---------- %
\begin{tabularx}{\linewidth}[ht]{@{}LR@{}}
	\multicolumn{2}{@{}l@{}}{\headbf{Compression}} \\
	\mytoprule
	\cmd{tar cf \vb{FILE.tar} \vb{FILES}}		& convert \vb{FILES} into \vb{FILE.tar} \hlx
	\cmd{tar xf \vb{FILE.tar}}					& extract files from \vb{FILE.tar} \hlx
	\cmd{tar czf \vb{FILE.tar.gz} \vb{FILES}}	& compress \vb{FILES} into \vb{FILE.tar.gz} using Gzip \hlx
	\cmd{tar xfz \vb{FILE.tar.gz}}				& extract files from \vb{FILE.tar.gz} using Gzip \hlx
	\cmd{gzip \vb{FILE}}						& compress \vb{FILE} and rename it as \vb{FILE.gz} \hlx
	\cmd{gzip -d \vb{FILE.gz}}					& decompress \vb{FILE.gz} back to \vb{FILE} \\
	\mybottomrule
\end{tabularx}
\spacebtwtables

% ---------- Network ---------- %
\begin{tabularx}{\linewidth}[ht]{@{}LR@{}}
	\multicolumn{2}{@{}l@{}}{\headbf{Network}} \\
	\mytoprule
	\cmd{ip address}			& print all internet protocol addresses \hlx
	\cmd{ping \vb{HOST}}		& ping \vb{HOST} and print results \hlx
	\cmd{whois \vb{DOMAIN}}		& print information about \vb{DOMAIN} \hlx
	\cmd{dig \vb{DOMAIN}}		& print DNS of \vb{DOMAIN} \hlx
	\cmd{dig -x \vb{HOST}}		& reverse lookup \vb{HOST} \hlx
	\cmd{wget \vb{FILE}}		& download \vb{FILE} \\
	\mybottomrule
\end{tabularx}
\spacebtwtables

% ---------- Terminator ---------- %
\begin{tabularx}{\linewidth}[ht]{@{}LR@{}}
	\multicolumn{2}{@{}l@{}}{\headbf{Terminator}} \\
	\mytoprule
	\cmd{\ctrlp \altp T}		& launch a new terminal \hlx
	\cmd{\ctrlp C}				& kill the current process \hlx
	\cmd{\ctrlp Z}				& suspend  the current process \\
	\cmd{fg}					& resume the suspended process in foreground \\
	\cmd{bg}					& resume the suspended process in background \hlx
	\cmd{\ctrlp D}				& log out of the current session \hlx
	\cmd{\ctrlp W}				& erase one word in the current line \hlx
	\cmd{\ctrlp U}				& erase the whole current line \hlx
	\cmd{\ctrlp R}				& reverse search in the previous commands \hlx
	\cmd{!!}					& execute the last command \hlx
	\cmd{exit}					& log out of the current session \hlx
	\cmd{\ctrlp \shiftp E}		& split the window vertically vertically \hlx
	\cmd{\ctrlp \shiftp O}		& split the window horizontally \\
	\mybottomrule
\end{tabularx}
\spacebtwtables

% ---------- Secure Shell ---------- %
\begin{tabularx}{\linewidth}[ht]{@{}LR@{}}
	\multicolumn{2}{@{}l@{}}{\headbf{Secure Shell (SSH)}} \\
	\mytoprule
	\cmd{ssh \vb{USER}@\vb{HOST}}					& connect to \vb{HOST} as \vb{USER} \hlx
	\cmd{ssh \vb{IP{\_}ADDRESS}}					& connect to \vb{IP{\_}ADDRESS} \hlx
	\cmd{ssh -p \vb{PORT} \vb{USER}@\vb{HOST}}		& connect to \vb{HOST} on \vb{PORT} as \vb{USER} \hlx
	\cmd{ssh-copy-id \vb{USER}@\vb{HOST}}			& add the key to \vb{HOST} as \vb{USER} \\
	\mybottomrule
\end{tabularx}
\spacebtwtables

% ---------- Package ---------- %
\begin{tabularx}{\linewidth}[ht]{@{}LR@{}}
	\multicolumn{2}{@{}l@{}}{\headbf{Package}} \\
	\mytoprule
	\cmd{apt-get update}					& synchronize package index files from sources \\
	\cmd{apt-get upgrade}					& install latest versions of installed packages \\
	\cmd{apt-get install \vb{PACKAGE}}		& install \vb{PACKAGE} \hlx
	\cmd{dpkg -i \vb{PACKAGE.deb}}			& install a Debian package \vb{PACKAGE.deb} \hlx
	\cmd{./configure}						& configure building settings \\
	\cmd{make}								& build the program from source code \\
	\cmd{make install}						& install the program \\
	\mybottomrule
\end{tabularx}
\spacebtwtables

% ---------- Searching ---------- %
\begin{tabularx}{\linewidth}[ht]{@{}LR@{}}
	\multicolumn{2}{@{}l@{}}{\headbf{Searching}} \\
	\mytoprule
	\cmd{grep \vb{PATTERN} \vb{FILE}}		& search for \vb{PATTERN} in \vb{FILE} \hlx
	\cmd{grep -r \vb{PATTERN} \vb{DIR}}		& recursively search for \vb{PATTERN} in \vb{DIR} \hlx
	\cmd{\vb{CMD} | grep \vb{PATTERN}}		& search for \vb{PATTERN} in \vb{CMD}'s output \hlx
	\cmd{locate \vb{FILENAME}}				& find all files whose name contain \vb{FILENAME} \\
	\mybottomrule
\end{tabularx}
\spacebtwtables

% ---------- Git ---------- %
\begin{tabularx}{\linewidth}[ht]{@{}LR@{}}
	\multicolumn{2}{@{}l@{}}{\headbf{Git}} \\
	\mytoprule
	\cmd{git clone \vb{URL}}			& clone the repository from \vb{URL} \hlx
	\cmd{git status}					& print current branch status \vb{BRANCH} \\
	\cmd{git branch \vb{BRANCH}}		& create a new branch named \vb{BRANCH} \\
	\cmd{git checkout \vb{BRANCH}}		& switch to the branch named \vb{BRANCH} \\
	\cmd{git merge \vb{BRANCH}}			& combine \vb{BRANCH} into the current one \hlx
	\cmd{git fetch}						& download all history from GitHub \\
	\cmd{git merge}						& combine remote branches into local branch \\
	\cmd{git push}						& upload all local branch commits to GitHub \\
	\cmd{git pull}						& update local branch from GitHub \hlx
	\cmd{git log}						& list version history for current branch \\
	\cmd{git log --follow \vb{FILE}}	& list version history for \vb{FILE} \\
	\cmd{git show \vb{COMMIT}}			& output content changes of \vb{COMMIT} \\
	\cmd{git add \vb{FILE}}				& stage \vb{FILE} \\
	\cmd{git commit -m "\vb{MESSAGE}"}	& commit staged file with \vb{MESSAGE} \hlx
	\cmd{git reset \vb{FILE}}			& reset \vb{FILE} \\
	\cmd{git reset --hard}				& reset all uncommitted changes \\
	\cmd{git clean -fd}					& recursively force remove unstaged files \\
	\mybottomrule
\end{tabularx}
\spacebtwtables

% ---------- Miscellaneous ---------- %
\begin{tabularx}{\linewidth}[ht]{@{}LR@{}}
	\multicolumn{1}{@{}l@{}}{\headbf{Miscellaneous}} \\
	\mytoprule
	Hitting \code{Tab} while typing a command, file name, and option will auto-complete it. \\
	\code{sudo} (superuser do) runs command with elevated privilege. \\
	\code{tar} (tape archive) deal with tape drives backup. \\
	Appending \code{--help} after a command will print command usage \\
	When simulating in ROS, remember \cmd{set use\_sim\_time true} and to append \code{{-}{-}clock}. \\
	\mybottomrule
\end{tabularx}
\newpage

% ---------- ROS Catkin Workspace ---------- %
\begin{tabularx}{\linewidth}[ht]{@{}LR@{}}
	\multicolumn{2}{@{}l@{}}{\headbf{ROS Catkin Workspace}} \\
	\mytoprule
	\cmd{roscd \vb{PACKAGE}}						& change directory to \vb{PACKAGE} \hlx
	\cmd{mkdir -p \textasciitilde/catkin\_ws/src}	& create a new catkin workspace \\
	\cmd{cd \textasciitilde/catkin\_ws}				& \\
	\cmd{catkin\_make}								& \hlx
	\cmd{catkin build}								& build the whole workspace \\
	\cmd{catkin build \vb{PACKAGE}}					& build \vb{PACKAGE} \\
	\cmd{catkin clean}								& clean the whole workspace \\
	\cmd{catkin clean \vb{PACKAGE}}					& clear \vb{PACKAGE} \\
	\mybottomrule
\end{tabularx}
\spacebtwtables

% ---------- ROS Run ---------- %
\begin{tabularx}{\linewidth}[ht]{@{}LS@{}}
	\multicolumn{2}{@{}l@{}}{\headbf{ROS Run}} \\
	\mytoprule
	\cmd{roscore}									& invoke the core of ros \hlx
	\cmd{rosrun \vb{PACKAGE} \vb{EXECUTABLE}}		& run \vb{EXECUTABLE} in \vb{PACKAGE} \hlx
	\cmd{roslaunch \vb{PACKAGE} \vb{LAUNCHFILE}}	& launch \vb{LAUNCHFILE} in \vb{PACKAGE} \\
	\mybottomrule
\end{tabularx}
\spacebtwtables

% ---------- ROS Node ---------- %
\begin{tabularx}{\linewidth}[ht]{@{}LR@{}}
	\multicolumn{2}{@{}l@{}}{\headbf{ROS Node}} \\
	\mytoprule
	\cmd{rosnode ping \vb{NODE}}		& test connectivity to \vb{NODE}\hlx
	\cmd{rosnode list}					& list active nodes \hlx
	\cmd{rosnode info \vb{NODE}}		& print information about \vb{NODE} \hlx
	\cmd{rosnode machine}				& list nodes running on the machine \hlx
	\cmd{rosnode kill \vb{NODE}}		& kill a running node \\
	\mybottomrule
\end{tabularx}
\spacebtwtables

% ---------- ROS Service ---------- %
\begin{tabularx}{\linewidth}[ht]{@{}LR@{}}
	\multicolumn{2}{@{}l@{}}{\headbf{ROS Service}} \\
	\mytoprule
	\cmd{rosservice list}							& list active services \hlx
	\cmd{rosservice call \vb{SERVICE} \vb{ARGS}}	& call \vb{SERVICE} with \vb{ARGS} \hlx
	\cmd{rosservice find \vb{TYPE}}					& find services with \vb{TYPE} \hlx
	\cmd{rosservice info \vb{SERVICE}}				& print information about \vb{SERVICE} \hlx
	\cmd{rosservice type \vb{SERVICE}}				& print type of \vb{SERVICE} \hlx
	\cmd{rosservice uri \vb{SERVICE}}				& print uri of \vb{SERVICE} \hlx
	\cmd{rossrv show \vb{TYPE}}						& print structure of \vb{TYPE} \\
	\mybottomrule
\end{tabularx}
\spacebtwtables

% ---------- ROS Topic ---------- %
\begin{tabularx}{\linewidth}[ht]{@{}LR@{}}
	\multicolumn{2}{@{}l@{}}{\headbf{ROS Topic}} \\
	\mytoprule
	\cmd{rostopic list}					& print information about active topics \hlx
	\cmd{rostopic bw \vb{TOPIC}}		& display bandwidth used by \vb{TOPIC} \hlx
	\cmd{rostopic echo \vb{TOPIC}}		& print messages from \vb{TOPIC}\hlx
	\cmd{rostopic find \vb{TYPE}}		& find topics with \vb{TYPE} \hlx
	\cmd{rostopic hz \vb{TOPIC}}		& display publishing rate of \vb{TOPIC} \hlx
	\cmd{rostopic info \vb{TOPIC}}		& print information about \vb{TOPIC} \hlx
	\cmd{rostopic pub \vb{TOPIC}}		& publish data to \vb{TOPIC} \hlx
	\cmd{rostopic type\vb{TOPIC}}		& print type of \vb{TOPIC} \hlx
	\cmd{rosmsg show \vb{TYPE}}			& print structure of \vb{TYPE} \\
	\mybottomrule
\end{tabularx}
\spacebtwtables

% ---------- ROS Parameter ---------- %
\begin{tabularx}{\linewidth}[ht]{@{}LR@{}}
	\multicolumn{2}{@{}l@{}}{\headbf{ROS Parameter}} \\
	\mytoprule
	\cmd{rosparam list}						& list all parameter names \hlx
	\cmd{rosparam set \vb{PARAM} \vb{VAL}}	& set \vb{PARAM} to \vb{VAL} \hlx
	\cmd{rosparam get \vb{PARAM}}			& print value of \vb{PARAM} \hlx
	\cmd{rosparam load \vb{YAML}}			& load parameters from \vb{YAML} \hlx
	\cmd{rosparam dump \vb{YAML}}			& dump parameters to \vb{YAML} \hlx
	\cmd{rosparam delete \vb{PARAM}}		& delete \vb{PARAM} \\
	\mybottomrule
\end{tabularx}
\spacebtwtables

% ---------- ROS Bag ---------- %
\begin{tabularx}{\linewidth}[ht]{@{}LR@{}}
	\multicolumn{2}{@{}l@{}}{\headbf{ROS Bag}} \\
	\mytoprule
	\cmd{rosbag record \vb{TOPIC}}		& record \vb{TOPIC} into bag \hlx
	\cmd{rosbag info \vb{BAG}}			& print content summary of \vb{BAG} \hlx
	\cmd{rosbag play \vb{BAG}}			& play back content of \vb{BAG} \hlx
	\cmd{rosbag check \vb{BAG}}			& check play-ability of \vb{BAG} in current system \hlx
	\cmd{rosbag compress \vb{BAG}}		& compress \vb{BAG} using BZ2 \hlx
	\cmd{rosbag decompress \vb{BAG}}	& decompress \vb{BAG} using BZ2 \\
	\mybottomrule
\end{tabularx}
\spacebtwtables

% ---------- ROS Packge Structure ---------- %
\begin{tabularx}{\linewidth}[ht]{@{}LR@{}}
	\multicolumn{2}{@{}l@{}}{\headbf{ROS Packge Structure}} \\
	\mytoprule
	\code{package.xml}		& manifest, dependencies and plugins \hlx
	\code{CMakeLists.txt}	& description of compilation procedure \hlx
	\code{src/}				& C and C++ source codes \hlx
	\code{build/}			& generated makefiles and support files \hlx
	\code{devel/}			& compiled binaries, libraries, headers \hlx
	\code{include/}			& C and C++ header files \hlx
	\code{scripts/}			& Python and bash scripts \hlx
	\code{config/}			& ymal configuration files \hlx
	\code{cfg/}				& dynamics reconfigure scripts \hlx
	\code{launch/}			& launch files \\
	\mybottomrule
\end{tabularx}
\spacebtwtables

% ---------- ROS Visualization Tools ---------- %
\begin{tabularx}{\linewidth}[ht]{@{}LR@{}}
	\multicolumn{2}{@{}l@{}}{\headbf{ROS Visualization Tools}} \\
	\mytoprule
	\cmd{rviz}				& 3D visualization of data and models \hlx
	\cmd{gzclient}			& Gazebo GUI \hlx
	\cmd{rqt}				& powerful GUI tool \\
	\cmd{rqt\_plot}			& simple and lightweight plotting \\
	\cmd{rqt\_bag}			& visualize content of a bag \\
	\cmd{rqt\_image\_view}	& visualize camera images \\
	\mybottomrule
\end{tabularx}

\vfill\null
\columnbreak

% ---------- TODO ---------- %
\begin{tabularx}{\linewidth}[ht]{@{}L@{}}
	\multicolumn{1}{@{}l@{}}{\headbf{TODO}} \\
	\mytoprule
	use ubuntumono font for commands \hlx
	is git reset unstage or reset? \hlx
	rosnode machine \hlx
	CMakeList \hlx
	Eigen \hlx
	ROS programming \\
	\mybottomrule
\end{tabularx}

\end{multicols*}
\end{document}