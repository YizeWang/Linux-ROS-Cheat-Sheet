\documentclass[a4paper]{article}

% import packages
\usepackage{multicol} % multiple column
\usepackage{booktabs} % table quality improvement
\usepackage{tabularx} % tabular package
\usepackage{mdframed} % for title
\usepackage{libertine} % libertine font
\usepackage[T1]{fontenc} % font encodings
\usepackage[table]{xcolor} % color manager
\usepackage{tikz} % shaded block
\usepackage[margin=1cm, landscape]{geometry} % document layout

% define new commands for shortcuts
\newcommand{\altp}{Alt$+$}
\newcommand{\ctrlp}{Ctrl$+$}
\newcommand{\shiftp}{Shift$+$}
\newcommand{\cmd}[1]{\textbf{#1}}
\newcommand{\hlx}{\\ \midrule[0.3ex]}
\newcommand{\mytoprule}{\toprule[0.5ex]}
\newcommand{\mybottomrule}{\bottomrule[0.5ex]}
\newcommand{\vb}[1]{\tikz[baseline]{\node[rectangle, rounded corners=0.5mm, text=black, fill=black!20, inner sep=1pt, anchor=base]{\$#1}}} % variable box

% document settings
\pagestyle{empty}
\setlength{\parindent}{0pt}
\setlength{\arrayrulewidth}{1pt}
\newcommand{\spacebtwtables}{\newline \vspace*{1em} \newline}

% title settings
\mdfdefinestyle{TitleStyle}{linecolor=gray, linewidth=1pt, leftmargin=0mm, rightmargin=0mm, skipbelow=0mm, skipabove=0mm}

% table settings
\renewcommand\arraystretch{1.2}
\renewcommand{\aboverulesep}{0pt}
\renewcommand{\belowrulesep}{0pt}
\newcommand{\headbf}[1]{\Large\textbf{#1}}
\newcolumntype{L}{>{\columncolor{white}[0pt][\tabcolsep]}X}
\newcolumntype{R}{>{\columncolor{white}[\tabcolsep][0pt]}m{20em}}
\newcolumntype{S}{>{\columncolor{white}[\tabcolsep][0pt]}m{16em}}

\begin{document}
\begin{multicols*}{3}

% ---------- Title ---------- %
\begin{mdframed}[style=TitleStyle]
	\begin{center}
		\fontfamily{pnc} \selectfont
		\large\scshape
		Linux and ROS \\
		Cheat Sheet \\
		\bigskip
		\footnotesize Author: Yize Wang \\
		\footnotesize Last Updated: \today
	\end{center}
\end{mdframed}

\scriptsize
\bigskip

\textbf{Variable Declaration} VARIABLE stands for a variable whose name is VARIABLE. For example, FILE means a file named FILE. \\
\textbf{sudo} sudo means super user do, elevated right granted \\
\spacebtwtables

% ---------- File Commands ---------- %
\begin{tabularx}{\linewidth}[ht]{@{}LR@{}}
	\multicolumn{2}{@{}l@{}}{\headbf{File Commands}} \\
	\mytoprule
	\cmd{ls}					& list contents of files and directories \hlx
	\cmd{ls -a}					& list hidden files and directories \hlx
	\cmd{cd \vb{DIR}}				& change working directory to \vb{DIR} \hlx
	\cmd{cd}					& change working directory to home \hlx
	\cmd{mkdir \vb{DIR}}			& create a directory named \vb{DIR} \hlx
	\cmd{pwd}					& print working directory \hlx
	\cmd{rm \vb{FILE}}				& remove \vb{FILE} \hlx
	\cmd{rm -r \vb{DIR}}			& remove \vb{DIR} \hlx
	\cmd{rm -f \vb{FILE}}			& force remove \vb{FILE} \hlx
	\cmd{rm -rf \vb{DIR}}			& force remove \vb{DIR} \hlx
	\cmd{cp \vb{FILE1} \vb{FILE2}}	& copy \vb{FILE1} to \vb{FILE2} \hlx
	\cmd{cp -r \vb{DIR1} \vb{DIR2}}	& copy \vb{DIR1} to \vb{DIR2} \hlx
	\cmd{mv \vb{FILE1} \vb{FILE2}}	& move \vb{FILE1} to \vb{FILE2} \hlx
	\cmd{ln -s \vb{FILE} \vb{LINK}}	& create symbolic link \vb{LINK} to \vb{FILE} \hlx
	\cmd{touch \vb{FILE}}			& create \vb{FILE} \hlx
	\cmd{cat \vb{FILE}}			& view content of \vb{FILE} \hlx
	\cmd{cat > \vb{FILE}}			& write input into \vb{FILE} \hlx
	\cmd{more \vb{FILE}}			& print contents of \vb{FIL}E \hlx
	\cmd{head \vb{FILE}}			& print the first 10 lines of \vb{FILE} \hlx
	\cmd{tail \vb{FILE}}			& print the last 10 lines of \vb{FILE} \\
	\mybottomrule
\end{tabularx}
\spacebtwtables

% ---------- System Info ---------- %
\begin{tabularx}{\linewidth}[ht]{@{}LR@{}}
	\multicolumn{2}{@{}l@{}}{\headbf{System Info}} \\
	\mytoprule
	\cmd{env}					& print environment variables \hlx
	\cmd{date}					& print system date and time \hlx
	\cmd{cal}					& print current month calendar \hlx
	\cmd{uptime}				& print system uptime \hlx
	\cmd{w}						& print online users \hlx
	\cmd{whoami}				& print current user \hlx
	\cmd{finger \vb{USER}}			& print information about \vb{USER} \hlx
	\cmd{uname -a}				& print kernel information \hlx
	\cmd{cat /proc/cpuinfo}		& print cpu information \hlx
	\cmd{cat /proc/meminfo}		& print memory information \hlx
	\cmd{man \vb{COMMAND}}			& print user manual of \vb{COMMAND} \hlx
	\cmd{df}					& print disk usage \hlx
	\cmd{du}					& print directory space usage \hlx
	\cmd{free}					& print memory and swap usage \hlx
	\cmd{whereis \vb{APP}}			& print locations of \vb{APP} \hlx
	\cmd{which \vb{APP}}			& print print executable file of \vb{APP} \\
	\mybottomrule
\end{tabularx}
\spacebtwtables

% ---------- Compression ---------- %
\begin{tabularx}{\linewidth}[ht]{@{}LR@{}}
	\multicolumn{2}{@{}l@{}}{\headbf{Compression}} \\
	\mytoprule
	\cmd{tar cf \vb{FILE.tar} \vb{FILES}}			& convert \vb{FILES} into \vb{FILE.tar} \hlx
	\cmd{tar xf \vb{FILE.tar}}					& extract files from \vb{FILE.tar} \hlx
	\cmd{tar czf \vb{FILE.tar.gz} \vb{FILES}}		& compress \vb{FILES} into \vb{FILE.tar.gz} using Gzip \hlx
	\cmd{tar xfz \vb{FILE.tar.gz}}				& extract files from \vb{FILE.tar.gz} using Gzip \hlx
	\cmd{gzip \vb{FILE}}						& compress \vb{FILE} and rename it as \vb{FILE.gz} \hlx
	\cmd{gzip -d \vb{FILE.gz}}					& decompress \vb{FILE.gz} back to \vb{FILE} \\
	\mybottomrule
\end{tabularx}
\spacebtwtables

% ---------- Network ---------- %
\begin{tabularx}{\linewidth}[ht]{@{}LR@{}}
	\multicolumn{2}{@{}l@{}}{\headbf{Network}} \\
	\mytoprule
	\cmd{ip address}					& print all internet protocol addresses \hlx
	\cmd{ping \vb{HOST}}					& ping \vb{HOST} and print results \hlx
	\cmd{whois \vb{DOMAIN}}				& print information about \vb{DOMAIN} \hlx
	\cmd{dig \vb{DOMAIN}}					& print DNS of \vb{DOMAIN} \hlx
	\cmd{dig -x \vb{HOST}}					& reverse lookup \vb{HOST} \hlx
	\cmd{wget \vb{FILE}}					& download \vb{FILE} \\
	\mybottomrule
\end{tabularx}
\spacebtwtables

% ---------- Terminator ---------- %
\begin{tabularx}{\linewidth}[ht]{@{}LR@{}}
	\multicolumn{2}{@{}l@{}}{\headbf{Terminator}} \\
	\mytoprule
	\cmd{\ctrlp \altp T}		& launch a new terminal \hlx
	\cmd{\ctrlp C}				& kill the current process \hlx
	\cmd{\ctrlp Z}				& suspend  the current process \\
	\cmd{fg}					& resume the suspended process in foreground \\
	\cmd{bg}					& resume the suspended process in background \hlx
	\cmd{\ctrlp D}				& log out of the current session \hlx
	\cmd{\ctrlp W}				& erase one word in the current line \hlx
	\cmd{\ctrlp U}				& erase the whole current line \hlx
	\cmd{\ctrlp R}				& reverse search in the previous commands \hlx
	\cmd{!!}					& execute the last command \hlx
	\cmd{exit}					& log out of the current session \hlx
	\cmd{\ctrlp \shiftp E}		& split the window vertically vertically \hlx
	\cmd{\ctrlp \shiftp O}		& split the window horizontally \\
	\mybottomrule
\end{tabularx}
\spacebtwtables

% ---------- Package ---------- %
\begin{tabularx}{\linewidth}[ht]{@{}LR@{}}
	\multicolumn{2}{@{}l@{}}{\headbf{Package}} \\
	\mytoprule
	\cmd{apt-get update}					& synchronize package index files from sources \\
	\cmd{apt-get upgrade}					& install latest versions of installed packages \\
	\cmd{apt-get install \vb{PACKAGE}}	& install \vb{PACKAGE} \hlx
	\cmd{dpkg -i \vb{PACKAGE.deb}}		& install a Debian package \vb{PACKAGE.deb} \hlx
	\cmd{./configure}						& configure building settings \\
	\cmd{make}								& build the program from source code \\
	\cmd{make install}						& install the program \\
	\mybottomrule
\end{tabularx}
\spacebtwtables

% ---------- Secure Shell ---------- %
\begin{tabularx}{\linewidth}[ht]{@{}LR@{}}
	\multicolumn{2}{@{}l@{}}{\headbf{Secure Shell (SSH)}} \\
	\mytoprule
	\cmd{ssh \vb{USER}@\vb{HOST}}					& connect to \vb{HOST} as \vb{USER} \hlx
	\cmd{ssh \vb{IP{\_}ADDRESS}}						& connect to \vb{IP{\_}ADDRESS} \hlx
	\cmd{ssh -p \vb{PORT} \vb{USER}@\vb{HOST}}	& connect to \vb{HOST} on \vb{PORT} as \vb{USER} \hlx
	\cmd{ssh-copy-id \vb{USER}@\vb{HOST}}			& add the key to \vb{HOST} as \vb{USER} \\
	\mybottomrule
\end{tabularx}
\spacebtwtables

% ---------- Searching ---------- %
\begin{tabularx}{\linewidth}[ht]{@{}LR@{}}
	\multicolumn{2}{@{}l@{}}{\headbf{Searching}} \\
	\mytoprule
	\cmd{grep \vb{PATTERN} \vb{FILE}}			& search for \vb{PATTERN} in \vb{FILE} \hlx
	\cmd{grep -r \vb{PATTERN} \vb{DIR}}			& recursively search for \vb{PATTERN} in \vb{DIR} \hlx
	\cmd{\vb{COMMAND} | grep \vb{PATTERN}}		& search for \vb{PATTERN} in \vb{COMMAND}'s output \hlx
	\cmd{locate \vb{FILE{\_}NAME}}				& find all files whose name contain \vb{FILE{\_}NAME} \\
	\mybottomrule
\end{tabularx}
\spacebtwtables

% ---------- Git ---------- %
\begin{tabularx}{\linewidth}[ht]{@{}LR@{}}
	\multicolumn{2}{@{}l@{}}{\headbf{Git}} \\
	\mytoprule
	\cmd{git clone \vb{URL}}			& clone the repository from \vb{URL} \hlx
	\cmd{git status}					& print current branch status \vb{BRANCH} \\
	\cmd{git branch \vb{BRANCH}}		& create a new branch named \vb{BRANCH} \\
	\cmd{git checkout \vb{BRANCH}}		& switch to the branch named \vb{BRANCH} \\
	\cmd{git merge \vb{BRANCH}}			& combine \vb{BRANCH} into the current one \hlx
	\cmd{git fetch}						& download all history from GitHub \\
	\cmd{git merge}						& combine remote branches into local branch \\
	\cmd{git push}						& upload all local branch commits to GitHub \\
	\cmd{git pull}						& update local branch from GitHub \hlx
	\cmd{git log}						& list version history for current branch \\
	\cmd{git log --follow \vb{FILE}}	& list version history for \vb{FILE} \\
	\cmd{git show \vb{COMMIT}}			& output content changes of \vb{COMMIT} \\
	\cmd{git add \vb{FILE}}				& stage \vb{FILE} \\
	\cmd{git commit -m "\vb{MESSAGE}"}	& commit staged file with \vb{MESSAGE} \hlx
	\cmd{git reset \vb{FILE}}			& reset \vb{FILE} \\
	\cmd{git reset --hard}				& reset all uncommitted changes \\
	\cmd{git clean -fd}					& recursively force remove unstaged files \\
	\mybottomrule
\end{tabularx}
\spacebtwtables

% ---------- .bashrc ---------- %
\begin{tabularx}{\linewidth}[ht]{@{}LR@{}}
	\multicolumn{2}{@{}l@{}}{\headbf{.bashrc}} \\
	\mytoprule
	\cmd{}			& \hlx
	\cmd{}			& \hlx
	\cmd{}			& \hlx
	\cmd{}			& \hlx
	\cmd{}			& \hlx
	\cmd{}			& \hlx
	\cmd{}			& \hlx
	\cmd{}			& \hlx
	\cmd{}			& \hlx
	\cmd{}			& \hlx
	\cmd{}			& \hlx
	\cmd{}			& \hlx
	\cmd{}			& \hlx
	\cmd{}			& \hlx
	\cmd{}			& \\
	\mybottomrule
\end{tabularx}
\spacebtwtables

% ---------- ROS Run ---------- %
\begin{tabularx}{\linewidth}[ht]{@{}LS@{}}
	\multicolumn{2}{@{}l@{}}{\headbf{ROS Run}} \\
	\mytoprule
	\cmd{roscore}									& invoke the core of ros \hlx
	\cmd{rosrun \vb{PACKAGE} \vb{EXECUTABLE}}		& run \vb{EXECUTABLE} in \vb{PACKAGE} \hlx
	\cmd{roslaunch \vb{PACKAGE} \vb{LAUNCHFILE}}	& launch \vb{LAUNCHFILE} in \vb{PACKAGE} \\
	\mybottomrule
\end{tabularx}
\spacebtwtables

% ---------- ROS Node ---------- %
\begin{tabularx}{\linewidth}[ht]{@{}LR@{}}
	\multicolumn{2}{@{}l@{}}{\headbf{ROS Node}} \\
	\mytoprule
	\cmd{rosnode ping \vb{NODE}}		& test connectivity to \vb{NODE}\hlx
	\cmd{rosnode list}					& list active nodes \hlx
	\cmd{rosnode info \vb{NODE}}		& print information about \vb{NODE} \hlx
	\cmd{rosnode machine}				& list nodes running on the machine \hlx
	\cmd{rosnode kill \vb{NODE}}		& kill a running node \\
	\mybottomrule
\end{tabularx}
\spacebtwtables

% ---------- ROS Package Structure ---------- %
\begin{tabularx}{\linewidth}[ht]{@{}LR@{}}
	\multicolumn{2}{@{}l@{}}{\headbf{ROS Package Structure}} \\
	\mytoprule
	\cmd{}			& \hlx
	\cmd{}			& \hlx
	\cmd{}			& \hlx
	\cmd{}			& \hlx
	\cmd{}			& \hlx
	\cmd{}			& \hlx
	\cmd{}			& \hlx
	\cmd{}			& \hlx
	\cmd{}			& \hlx
	\cmd{}			& \hlx
	\cmd{}			& \hlx
	\cmd{}			& \hlx
	\cmd{}			& \hlx
	\cmd{}			& \hlx
	\cmd{}			& \\
	\mybottomrule
\end{tabularx}
\spacebtwtables

% ---------- ROS Visualization ---------- %
\begin{tabularx}{\linewidth}[ht]{@{}LR@{}}
	\multicolumn{2}{@{}l@{}}{\headbf{ROS Visualization}} \\
	\mytoprule
	\cmd{}			& \hlx
	\cmd{}			& \hlx
	\cmd{}			& \hlx
	\cmd{}			& \hlx
	\cmd{}			& \hlx
	\cmd{}			& \hlx
	\cmd{}			& \hlx
	\cmd{}			& \hlx
	\cmd{}			& \hlx
	\cmd{}			& \hlx
	\cmd{}			& \hlx
	\cmd{}			& \hlx
	\cmd{}			& \hlx
	\cmd{}			& \hlx
	\cmd{}			& \\
	\mybottomrule
\end{tabularx}
\spacebtwtables

% ---------- TODO ---------- %
\begin{tabularx}{\linewidth}[ht]{@{}L@{}}
	\multicolumn{1}{@{}l@{}}{\headbf{TODO}} \\
	\mytoprule
	ubuntu mono font for commands \hlx
	tab auto-completion \hlx
	tar explanation \\
	is git reset unstage or reset? \\
	simtime true\\
	copyright\\
	rosnode machine \\
	\mybottomrule
\end{tabularx}
\spacebtwtables

\end{multicols*}
\end{document}