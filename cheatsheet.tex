\documentclass[a4paper]{article}

% import packages
\usepackage{multicol} % multiple column
\usepackage{booktabs} % table quality improvement
\usepackage{tabularx} % tabular package
\usepackage{mdframed} % for title
\usepackage{libertine} % libertine font
\usepackage[T1]{fontenc} % font encoding

\usepackage[dvipsnames, table]{xcolor} % color manager
\usepackage[margin=1cm, landscape]{geometry} % document layout

% define new commands for shortcuts
\newcommand{\altp}{Alt$+$}
\newcommand{\ctrlp}{Ctrl$+$}
\newcommand{\shiftp}{Shift$+$}
\newcommand{\cmd}[1]{\textbf{#1}}
\newcommand{\hlx}{\\ \midrule[0.3ex]}
\newcommand{\mytoprule}{\toprule[0.5ex]}
\newcommand{\mybottomrule}{\bottomrule[0.5ex]}

% document settings
\pagestyle{empty}
\setlength{\parindent}{0pt}
\setlength{\arrayrulewidth}{1pt}
\newcommand{\spacebtwtables}{\newline \vspace*{1em} \newline}

% title settings
\mdfdefinestyle{TitleStyle}{linecolor=gray, linewidth=1pt, leftmargin=0mm, rightmargin=0mm, skipbelow=0mm, skipabove=0mm}

% table settings
\renewcommand\arraystretch{1.2}
\renewcommand{\aboverulesep}{0pt}
\renewcommand{\belowrulesep}{0pt}
\newcommand{\headbf}[1]{\Large\textbf{#1}}
\newcolumntype{L}{>{\columncolor{white}[0pt][\tabcolsep]}X}
\newcolumntype{R}{>{\columncolor{white}[\tabcolsep][0pt]}m{20em}}

\begin{document}
\begin{multicols*}{3}
	
% ---------- Title ---------- %
\begin{mdframed}[style=TitleStyle]
\begin{center}
	\fontfamily{pnc} \selectfont
	\large\scshape
	Linux and ROS \\
	Cheat Sheet \\
	\bigskip
	\footnotesize Author: Yize Wang \\
	\footnotesize Last Updated: \today
\end{center}
\end{mdframed}

\scriptsize
\bigskip

\textbf{Variable Declaration} \$VARIABLE stands for a variable whose name is VARIABLE. For example, \$FILE means a file named FILE. \\
\textbf{sudo} sudo means super user do, elevated right granted \\
\spacebtwtables

% ---------- File Commands ---------- %
\begin{tabularx}{\linewidth}[ht]{@{}LR@{}}
	\multicolumn{2}{@{}l@{}}{\headbf{File Commands}} \\
	\mytoprule
	\cmd{ls}					& list contents of files and directories \hlx
	\cmd{ls -a}					& list hidden files and directories \hlx
	\cmd{cd \$DIR}				& change working directory to \$DIR \hlx
	\cmd{cd}					& change working directory to home \hlx
	\cmd{mkdir \$DIR}			& create a directory named \$DIR \hlx
	\cmd{pwd}					& print working directory \hlx
	\cmd{rm \$FILE}				& remove \$FILE \hlx
	\cmd{rm -r \$DIR}			& remove \$DIR \hlx
	\cmd{rm -f \$FILE}			& force remove \$FILE \hlx
	\cmd{rm -rf \$DIR}			& force remove \$DIR \hlx
	\cmd{cp \$FILE1 \$FILE2}	& copy \$FILE1 to \$FILE2 \hlx
	\cmd{cp -r \$DIR1 \$DIR2}	& copy \$DIR1 to \$DIR2 \hlx
	\cmd{mv \$FILE1 \$FILE2}	& move \$FILE1 to \$FILE2 \hlx
	\cmd{ln -s \$FILE \$LINK}	& create symbolic link \$LINK to \$FILE \hlx
	\cmd{touch \$FILE}			& create \$FILE\hlx
	\cmd{cat \$FILE}			& view content of \$FILE \hlx
	\cmd{cat > \$FILE}			& write input into \$FILE \hlx
	\cmd{more \$FILE}			& print contents of \$FILE \hlx
	\cmd{head \$FILE}			& print the first 10 lines of \$FILE \hlx
	\cmd{tail \$FILE}			& print the last 10 lines of \$FILE \\
	\mybottomrule
\end{tabularx}
\spacebtwtables

% ---------- System Info ---------- %
\begin{tabularx}{\linewidth}[ht]{@{}LR@{}}
	\multicolumn{2}{@{}l@{}}{\headbf{System Info}} \\
	\mytoprule
	\cmd{env}					& print environment variables \hlx
	\cmd{date}					& print system date and time \hlx
	\cmd{cal}					& print current month calendar \hlx
	\cmd{uptime}				& print system uptime \hlx
	\cmd{w}						& print online users \hlx
	\cmd{whoami}				& print current user \hlx
	\cmd{finger \$USER}			& print information about \$USER \hlx
	\cmd{uname -a}				& print kernel information \hlx
	\cmd{cat /proc/cpuinfo}		& print cpu information \hlx
	\cmd{cat /proc/meminfo}		& print memory information \hlx
	\cmd{man \$COMMAND}			& print user manual of \$COMMAND \hlx
	\cmd{df}					& print disk usage \hlx
	\cmd{du}					& print directory space usage \hlx
	\cmd{free}					& print memory and swap usage \hlx
	\cmd{whereis \$APP}			& print locations of \$APP \hlx
	\cmd{which \$APP}			& print print executable file of \$APP \\
	\mybottomrule
\end{tabularx}
\spacebtwtables

% ---------- Compression ---------- %
\begin{tabularx}{\linewidth}[ht]{@{}LR@{}}
	\multicolumn{2}{@{}l@{}}{\headbf{Compression}} \\
	\mytoprule
	\cmd{tar cf \$FILE.tar \$FILES}			& convert \$FILES into \$FILE.tar \hlx
	\cmd{tar xf \$FILE.tar}					& extract files from \$FILE.tar \hlx
	\cmd{tar czf \$FILE.tar.gz \$FILES}		& compress \$FILES into \$FILE.tar.gz using Gzip \hlx
	\cmd{tar xfz \$FILE.tar.gz}				& extract files from \$FILE.tar.gz using Gzip \hlx
	\cmd{gzip \$FILE}						& compress \$FILE and rename it as \$FILE.gz \hlx
	\cmd{gzip -d \$FILE.gz}					& decompress \$FILE.gz back to \$FILE \\
	\mybottomrule
\end{tabularx}
\spacebtwtables

% ---------- Network ---------- %
\begin{tabularx}{\linewidth}[ht]{@{}LR@{}}
	\multicolumn{2}{@{}l@{}}{\headbf{Network}} \\
	\mytoprule
	\cmd{apt-get update}				& synchronize package index files from sources \hlx
	\cmd{apt-get upgrade}				& install latest versions of installed packages \hlx
	\cmd{apt-get install \$PACKAGE}		& install \$PACKAGE \hlx
	\cmd{ip address}					& print all internet protocol addresses \hlx
	\cmd{ping \$HOST}					& ping \$HOST and print results \hlx
	\cmd{whois \$DOMAIN}				& print information about \$DOMAIN \hlx
	\cmd{dig \$DOMAIN}					& print DNS of \$DOMAIN \hlx
	\cmd{dig -x \$HOST}					& reverse lookup \$HOST \hlx
	\cmd{wget \$FILE}					& download \$FILE \\
	\mybottomrule
\end{tabularx}
\spacebtwtables

% ---------- TODO ---------- %
\begin{tabularx}{\linewidth}[ht]{@{}L@{}}
	\multicolumn{1}{@{}l@{}}{\headbf{TODO}} \\
	\mytoprule
	difference between ls -a and ls -al \hlx
	code box for commands \hlx
	ubuntu mono font for commands \hlx
	tar explanation \\
	\mybottomrule
\end{tabularx}
\spacebtwtables


\end{multicols*}
\end{document}