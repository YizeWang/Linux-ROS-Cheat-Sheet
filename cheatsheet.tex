\documentclass[a4paper]{article}

% import packages
\usepackage{multicol} % multiple column
\usepackage{booktabs} % table quality improvement
\usepackage{tabularx} % tabular package
\usepackage{mdframed} % for title
\usepackage{libertine} % libertine font
\usepackage[T1]{fontenc} % font encodings
\usepackage[table]{xcolor} % color manager
\usepackage{tikz} % shaded block
\usepackage[margin=1cm, landscape]{geometry} % document layout

% define new commands for shortcuts
\newcommand{\altp}{Alt$+$}
\newcommand{\ctrlp}{Ctrl$+$}
\newcommand{\shiftp}{Shift$+$}
\newcommand{\prefixp}{Prefix$+$}
\newcommand{\cmt}[1]{\# #1}
\newcommand{\code}[1]{\textbf{#1}}
\newcommand{\cmd}[1]{\$ \textbf{#1}}
\newcommand{\tag}[1]{<#1>}
\newcommand{\doubledash}{{-}{-}}
\newcommand{\xmltag}[1]{<#1/>}
\newcommand{\hlx}{\\ \midrule[0.3ex]}
\newcommand{\mytoprule}{\toprule[0.5ex]}
\newcommand{\mybottomrule}{\bottomrule[0.5ex]}
\newcommand{\white}[1]{{\textcolor{white}{#1}}}
\newcommand{\cb}[1]{\tikz[baseline]{\node[rectangle, rounded corners=0.5mm, text=black, fill=black!20, inner sep=1pt, anchor=base]{#1}}} % code box
\newcommand{\vb}[1]{\tikz[baseline]{\node[rectangle, rounded corners=0.5mm, text=black, fill=black!20, inner sep=1pt, anchor=base]{\$#1}}} % variable box

% document settings
\pagestyle{empty}
\setlength{\parindent}{0pt}
\setlength{\arrayrulewidth}{1pt}

% title settings
\mdfdefinestyle{TitleStyle}{linecolor=gray, linewidth=1pt, leftmargin=0mm, rightmargin=0mm, skipbelow=0mm, skipabove=0mm}

% table settings
\renewcommand\arraystretch{1.2}
\renewcommand{\aboverulesep}{0pt}
\renewcommand{\belowrulesep}{0pt}
\newcommand{\headbf}[1]{\Large\textbf{#1}}
\newcolumntype{T}{>{\columncolor{white}}m{2em}}
\newcolumntype{L}{>{\columncolor{white}[0pt][\tabcolsep]}X}
\newcolumntype{R}{>{\columncolor{white}[\tabcolsep][0pt]}m{19em}}
\newcolumntype{B}{>{\columncolor{white}[\tabcolsep][0pt]}m{25em}}
\newcolumntype{S}{>{\columncolor{white}[\tabcolsep][0pt]}m{16em}}

\begin{document}
\begin{multicols*}{3}

% ---------- Title ---------- %
\begin{mdframed}[style=TitleStyle]
\begin{center}
    \medskip
    \fontfamily{pnc} \selectfont
    \large\scshape
    Linux \& ROS Cheat Sheet \\
    \medskip
    \footnotesize Author: Yize Wang \\
    \footnotesize Institute: Autonomous Systems Lab \\
    \footnotesize Last Updated: \today
\end{center}
\end{mdframed}

\scriptsize

% ---------- File Commands ---------- %
\begin{tabularx}{\linewidth}[ht]{@{}LR@{}}
    \multicolumn{2}{@{}l@{}}{\headbf{File Commands}} \\
    \mytoprule
    \cmd{ls}                                    & list contents of the current directory \\
    \cmd{ls -al}                                & list hidden contents of the current directory \hlx
    \cmd{cd}                                    & change the directory to home \\
    \cmd{cd -}                                  & change the directory to the previous one \\
    \cmd{cd \vb{DIR}}                           & change the directory to \vb{DIR} \hlx
    \cmd{mkdir \vb{DIR}}                        & make a new directory named \vb{DIR} \hlx
    \cmd{pwd}                                   & print the working directory \hlx
    \cmd{rm \vb{FILE}}                          & remove \vb{FILE} \\
    \cmd{rm -r \vb{DIR}}                        & remove \vb{DIR} recursively \\
    \cmd{rm -f \vb{FILE}}                       & force remove \vb{FILE} \\
    \cmd{rm -rf \vb{DIR}}                       & force remove \vb{DIR} recursively \hlx
    \cmd{cp \vb{FILE1} \vb{FILE2}/\vb{DIR}}     & copy \vb{FILE1} to \vb{FILE2}/\vb{DIR} \\
    \cmd{cp -r \vb{DIR1} \vb{DIR2}}             & copy \vb{DIR1} to \vb{DIR2} recursively \hlx
    \cmd{mv \vb{FILE1} \vb{FILE2}/\vb{DIR}}     & move \vb{FILE1} to \vb{FILE2}/\vb{DIR} \hlx
    \cmd{ln -s \vb{FILE} \vb{LINK}}             & create a symbolic link \vb{LINK} to \vb{FILE} \hlx
    \cmd{touch \vb{FILE}}                       & create \vb{FILE} \hlx
    \cmd{cat \vb{FILE}}                         & view content of \vb{FILE} \\
    \cmd{cat > \vb{FILE}}                       & write input into \vb{FILE} \hlx
    \cmd{echo \vb{STRING}/\vb{VAR}}             & print \vb{STRING}/value of \vb{VAR} \\
    \cmd{more \vb{FILE}}                        & print content of \vb{FILE} \\
    \cmd{head \vb{FILE}}                        & print the first 10 lines of \vb{FILE} \\
    \cmd{tail \vb{FILE}}                        & print the last 10 lines of \vb{FILE} \hlx
    \cmd{gedit \vb{FILE}}                       & edit \vb{FILE} using GUI text editor \\
    \cmd{vim \vb{FILE}}                         & edit \vb{FILE} using Vim \\
    \mybottomrule
\end{tabularx}
\bigskip

% ---------- System Info ---------- %
\begin{tabularx}{\linewidth}[ht]{@{}L@{}}
    { \begin{tabularx}{\linewidth}[ht]{@{}LR@{}}
        \multicolumn{2}{@{}l@{}}{\headbf{System Information}} \\
        \mytoprule
        \cmd{env}                   & print environment variables \hlx
        \cmd{date}                  & print system date and time \hlx
        \cmd{man \vb{COMMAND}}      & print user manual of \vb{COMMAND} \hlx
        \cmd{whereis \vb{APP}}      & print locations of \vb{APP} \hlx
        \cmd{which \vb{APP}}        & print executable file of \vb{APP} \hlx
        \cmd{ps}                    & print process status \\
        \cmd{ps -aux}               & print all running process \hlx
        \cmd{htop}                  & print currently running processes and more \hlx
    \end{tabularx} } \\
    { \begin{tabularx}{\linewidth}[ht]{@{}LTR@{}}
        path symbolic links         &\code{{.}}                 & current directory \\
                                    &\code{{.}{.}}              & parent directory \\
                                    &\code{\textasciitilde}     & home directory \\
                                    &\code{/}                   & root directory \hlx
        output direction            &\code{{>}}                 & to a file (rewrite) \\
                                    &\code{{>}{>}}              & to a file (append) \\
                                    &\code{{|}}                 & pipe  output of first command to second \\
        \mybottomrule
    \end{tabularx} } \\
\end{tabularx}

\vspace{0pt}
\columnbreak
\vspace{0pt}

% ---------- Linux Shell ---------- %
\begin{tabularx}{\linewidth}[ht]{@{}LR@{}}
    \multicolumn{2}{@{}l@{}}{\headbf{Linux Shell}} \\
    \mytoprule
    \code{\ctrlp C}             & kill the current process \hlx
    \code{\ctrlp Z}             & suspend  the current process \\
    \cmd{fg}                    & resume the suspended process in foreground \\
    \cmd{bg}                    & resume the suspended process in background \hlx
    \code{\ctrlp W}             & erase one word in the current line \hlx
    \code{\ctrlp U}             & erase the whole current line \hlx
    \code{\ctrlp R}             & reverse search in the previous commands \hlx
    \code{\ctrlp A}             & go to the beginning of the line \hlx
    \code{\ctrlp E}             & go to the end of the line \hlx
    \code{\ctrlp D}             & log out of the current session \hlx
    \cmd{exit}                  & log out of the current session \hlx
    \cmd{clear}                 & clear the terminal screen \\
    \mybottomrule
\end{tabularx}
Use \code{\ctrlp R} to reverse search, type part of a command and hit \code{\ctrlp R} repeatedly. \\
\code{\ctrlp A} is especially useful when you forget to add \cb{sudo} before the command.
\bigskip

% ---------- Git ---------- %
\begin{tabularx}{\linewidth}[ht]{@{}LR@{}}
    \multicolumn{2}{@{}l@{}}{\headbf{Git}} \\
    \mytoprule
    \cmd{git clone \vb{URL}}                & clone the repository from \vb{URL} \hlx
    \cmd{git status}                        & print current branch status \vb{BRANCH} \\
    \cmd{git branch \vb{BRANCH}}            & create a new branch named \vb{BRANCH} \\
    \cmd{git checkout \vb{BRANCH}}          & switch to the branch named \vb{BRANCH} \\
    \cmd{git merge \vb{BRANCH}}             & combine \vb{BRANCH} into the current one \hlx
    \cmd{git fetch}                         & download all history from GitHub \\
    \cmd{git merge}                         & combine remote branches into local branch \\
    \cmd{git push}                          & upload all local branch commits to GitHub \\
    \cmd{git pull}                          & update local branch from GitHub \hlx
    \cmd{git log}                           & list version history for current branch \\
    \cmd{git log {-}{-}follow \vb{FILE}}    & list version history for \vb{FILE} \\
    \cmd{git show \vb{COMMIT}}              & output content changes of \vb{COMMIT} \\
    \cmd{git add \vb{FILE}}                 & stage \vb{FILE} \\
    \cmd{git commit -m "\vb{MESSAGE}"}      & commit staged file with \vb{MESSAGE} \hlx
    \cmd{git reset \vb{FILE}}               & reset \vb{FILE} \\
    \cmd{git reset {-}{-}hard}              & reset all uncommitted changes \\
    \cmd{git clean -fd}                     & recursively force remove unstaged files \\
    \mybottomrule
\end{tabularx}
\bigskip

% ---------- Secure Shell ---------- %
\begin{tabularx}{\linewidth}[ht]{@{}LR@{}}
    \multicolumn{2}{@{}l@{}}{\headbf{Secure Shell (SSH)}} \\
    \mytoprule
    \cmd{ssh \vb{USER}@\vb{HOST}}                   & connect \vb{HOST} as \vb{USER} \hlx
    \cmd{ssh \vb{IP{\_}ADDRESS}}                    & connect \vb{IP{\_}ADDRESS} \hlx
    \cmd{ssh -p \vb{PORT} \vb{USER}@\vb{HOST}}      & connect \vb{HOST} on \vb{PORT} as \vb{USER} \hlx
    \cmd{ssh-copy-id \vb{USER}@\vb{HOST}}           & add the key to \vb{HOST} as \vb{USER} \\
    \mybottomrule
\end{tabularx}
\bigskip

% ---------- Package ---------- %
\begin{tabularx}{\linewidth}[ht]{@{}LR@{}}
    \multicolumn{2}{@{}l@{}}{\headbf{Package}} \\
    \mytoprule
    \cmd{sudo apt-get update}                    & synchronize package index files from sources \\
    \cmd{sudo apt-get upgrade}                   & install latest versions of installed packages \\
    \cmd{sudo apt-get install \vb{PACKAGE}}      & install \vb{PACKAGE} \hlx
    \cmd{sudo dpkg -i \vb{PACKAGE.deb}}          & install a Debian package \vb{PACKAGE.deb} \hlx
    \cmd{./configure}                            & configure building settings \\
    \cmd{make}                                   & build the program from source code \\
    \cmd{make install}                           & install the program \\
    \mybottomrule
\end{tabularx}

\vspace{0pt}
\columnbreak
\vspace{0pt}

% ---------- Terminator ---------- %
\begin{tabularx}{\linewidth}[ht]{@{}LR@{}}
    \multicolumn{2}{@{}l@{}}{\headbf{Terminator}} \\
    \mytoprule
    \code{\ctrlp \shiftp I}         & open a new window \hlx
    \code{\ctrlp \shiftp T}         & open a new tab \hlx
    \code{\ctrlp \shiftp E}         & split terminals vertically \hlx
    \code{\ctrlp \shiftp O}         & split terminals horizontally \hlx
    \code{\altp \tag{arrow key}}    & switch to a different terminal \\
    \mybottomrule
\end{tabularx}
\bigskip

% ---------- Terminal Multiplexer (TMUX) ---------- %
\begin{tabularx}{\linewidth}[ht]{@{}LR@{}}
    \multicolumn{2}{@{}l@{}}{\headbf{Terminal Multiplexer (TMUX)}} \\
    \mytoprule
    \cmd{tmux}                              & start TMUX\\
    \cmd{tmux ls}                           & list all sessions \\
    \cmd{tmux a -t \vb{SESSION\_NAME}}      & attach to \vb{SESSION\_NAME}\\
    \cmd{tmux new -s [\vb{SESSION\_NAME}]}  & create a new session with \vb{SESSION\_NAME} \hlx
    \code{\ctrlp B}                         & prefix \\
    \code{\prefixp \%}                      & split terminals horizontally \\
    \code{\prefixp "}                       & split terminals vertically \\
    \code{\prefixp \tag{arrow key}}         & switch to a different terminal \\
    \code{\prefixp C}                       & create a new window in current session \\
    \code{\prefixp \vb{NUM}}                & switch to \vb{NUM} window \\
    \code{\prefixp D}                       & detach from the current session \\
    \mybottomrule
\end{tabularx}
\bigskip

% ---------- Searching ---------- %
\begin{tabularx}{\linewidth}[ht]{@{}LR@{}}
    \multicolumn{2}{@{}l@{}}{\headbf{Searching}} \\
    \mytoprule
    \cmd{grep \vb{PATTERN} \vb{FILES}}      & search for \vb{PATTERN} in \vb{FILES} \\
    \cmd{grep -r \vb{PATTERN} \vb{DIR}}     & search for \vb{PATTERN} recursively in \vb{DIR} \\
    \cmd{grep -n \vb{PATTERN} \vb{FILES}}   & search for \vb{PATTERN} and print line numbers \\
    \cmd{grep -C1 \vb{PATTERN} \vb{FILES}}  & search for \vb{PATTERN} and print 1-line context \hlx
    \cmd{\vb{CMD} | grep \vb{PATTERN}}      & search for \vb{PATTERN} in \vb{CMD}'s output \hlx
    \cmd{sudo updatedb}                     & update searching database for \code{locate} command \\
    \cmd{locate -b \vb{PATTERN}}            & find files and dirs containing \vb{PATTERN} \\
    \mybottomrule
\end{tabularx}
\bigskip

% ---------- Docker ---------- %
\begin{tabularx}{\linewidth}[ht]{@{}L@{}}
    { \begin{tabularx}{\linewidth}[ht]{@{}LR@{}}
            \multicolumn{2}{@{}l@{}}{\headbf{Docker}} \\
            \mytoprule
            \cmd{docker build -t \vb{IMAGE}:\vb{TAG}}         & build \vb{IMAGE} with tag \vb{TAG} \\
            \cmd{docker image ls}                             & list all local images with Docker Engine \\
            \cmd{docker image rm \vb{IMAGE}:\vb{TAG}}         & delete image from local image store \hlx
    \end{tabularx} } \\
    { \begin{tabularx}{\linewidth}[ht]{@{}LR@{}}
            \cmd{docker tag \vb{OLD\_IMAGE}:\vb{OLD\_TAG} \vb{REGISTRY}/\vb{NEW\_IMAGE}:\vb{NEW\_TAG}} \\
            retag local image with new image name and tag \\
            Eg. \$ docker tag myimage:1.0 myrepo/myimage:2.0 \hlx
            \cmd{docker push \vb{REGISTRT}/\vb{IMAGE}:\vb{TAG}} \\
            push image to registry \\
            Eg. \$ docker push myrepo/myimage:2.0 \hlx
            Eg. \$ docker container run --name web -p 5000:80 alpine:3.9 \\
            run container from Alpine version 3.9 image, name the running container "web" and expose port 5000 externally, mapped to port 80 inside the container \hlx
            Eg. \$ docker container stop/kill web \\
            stop "web" container through SIGTERM/SIGKILL \\
            \mybottomrule
    \end{tabularx} }
\end{tabularx}
\bigskip

% ---------- Miscellaneous ---------- %
\begin{tabularx}{\linewidth}[ht]{@{}LR@{}}
    \multicolumn{2}{@{}l@{}}{\headbf{Miscellaneous}} \\
    \mytoprule
    \cmd{sudo \vb{COMMAND}}             & run \vb{COMMAND} with elevated privilege \hlx
    \cmd{\vb{COMMAND} {-}{-}help}       & print \vb{COMMAND}'s usage help \hlx
    \cmd{ip address}                    & print all internet protocol addresses \hlx
    \cmd{ping \vb{HOST}}                & ping \vb{HOST} and print results \hlx
    \cmd{tar xfz \vb{FILE.tar.gz}}      & extract files from \vb{FILE.tar.gz} \\
    \mybottomrule
\end{tabularx}

\vfill\null
\newpage

% ---------- ROS Catkin Workspace ---------- %
\begin{tabularx}{\linewidth}[ht]{@{}LR@{}}
    \multicolumn{2}{@{}l@{}}{\headbf{ROS Catkin Workspace}} \\
    \mytoprule
    \cmd{roscd \vb{PACKAGE}}           & change directory to \vb{PACKAGE}'s location \hlx
    \cmd{catkin build}                 & build the whole workspace \\
    \cmd{catkin build \vb{PACKAGE}}    & build \vb{PACKAGE} \\
    \cmd{catkin clean}                 & clean the whole workspace \\
    \cmd{catkin config \vb{OPTIONS}}   & configure catkin workspace with \vb{OPTIONS} \hlx
    \cmd{wstool init}                  & set up current directory as workspace \\
    \cmd{wstool merge \vb{ROSINSTALL}} & merge \vb{ROSINSTALL} into the workspace \\
    \cmd{wstool up}                    & update configuration elements \\
    \mybottomrule
\end{tabularx}
Always remember to \cb{\cmd{source \textasciitilde/catkin\_ws/devel/setup.bash}}.
\bigskip

% ---------- ROS Run ---------- %
\begin{tabularx}{\linewidth}[ht]{@{}L@{}}
    { \begin{tabularx}{\linewidth}[ht]{@{}LS@{}}
        \multicolumn{2}{@{}l@{}}{\headbf{ROS Run}} \\
        \mytoprule
        \cmd{roscore}                                   & invoke the core of ROS \hlx
        \cmd{roslaunch \vb{PACKAGE} \vb{LAUNCHFILE}}    & launch \vb{LAUNCHFILE} in \vb{PACKAGE} \hlx
    \end{tabularx} } \\
    { \begin{tabularx}{\linewidth}[ht]{@{}L@{}}
        \cmd{rosrun \vb{PACKAGE} \vb{EXECUTABLE} [\vb{PARAM}:=\vb{VALUE}]} \\
        \mbox{\phantom{\cmd{}}}run node \vb{EXECUTABLE} from \vb{PACKAGE} [with \vb{PARAM} set to \vb{VALUE}] \\
        Eg. \$ rosrun rviz rviz -d maplab.rviz \\
        \mybottomrule
    \end{tabularx} } \\
\end{tabularx}
\bigskip

% ---------- ROS Node ---------- %
\begin{tabularx}{\linewidth}[ht]{@{}LR@{}}
    \multicolumn{2}{@{}l@{}}{\headbf{ROS Node}} \\
    \mytoprule
    \cmd{rosnode ping \vb{NODE}}             & test connectivity to \vb{NODE}\hlx
    \cmd{rosnode list}                       & list active nodes \hlx
    \cmd{rosnode info \vb{NODE}}             & print information about \vb{NODE} \hlx
    \cmd{rosnode machine}                    & list nodes running on the machine \hlx
    \cmd{rosnode kill \vb{NODE}}             & kill the running \vb{NODE} \\
    \mybottomrule
\end{tabularx}
\bigskip

% ---------- ROS Parameter ---------- %
\begin{tabularx}{\linewidth}[ht]{@{}LR@{}}
    \multicolumn{2}{@{}l@{}}{\headbf{ROS Parameter}} \\
    \mytoprule
    \cmd{rosparam list}                      & list all parameter names \hlx
    \cmd{rosparam set \vb{PARAM} \vb{VAL}}   & set value of \vb{PARAM} to \vb{VAL} \hlx
    \cmd{rosparam get \vb{PARAM}}            & print value of \vb{PARAM} \hlx
    \cmd{rosparam load \vb{YAML}}            & load parameters from \vb{YAML} \hlx
    \cmd{rosparam dump \vb{YAML}}            & dump parameters to \vb{YAML} \hlx
    \cmd{rosparam delete \vb{PARAM}}         & delete \vb{PARAM} \\
    \mybottomrule
\end{tabularx}
\bigskip

% ---------- ROS Topic ---------- %
\begin{tabularx}{\linewidth}[ht]{@{}LR@{}}
    \multicolumn{2}{@{}l@{}}{\headbf{ROS Topic}} \\
    \mytoprule
    \cmd{rostopic list}                      & print information about active topics \hlx
    \cmd{rostopic bw \vb{TOPIC}}             & display bandwidth used by \vb{TOPIC} \hlx
    \cmd{rostopic echo \vb{TOPIC}}           & print messages from \vb{TOPIC}\hlx
    \cmd{rostopic find \vb{TYPE}}            & find topics of \vb{TYPE} \hlx
    \cmd{rostopic hz \vb{TOPIC}}             & display publishing rate of \vb{TOPIC} \hlx
    \cmd{rostopic info \vb{TOPIC}}           & print information about \vb{TOPIC} \hlx
    \cmd{rostopic pub \vb{TOPIC}}            & publish data to \vb{TOPIC} \hlx
    \cmd{rostopic type \vb{TOPIC}}           & print type of \vb{TOPIC} \hlx
    \cmd{rosmsg show \vb{TYPE}}              & print structure of \vb{TYPE} \\
    \mybottomrule
\end{tabularx}

\vspace{0pt}
\columnbreak
\vspace{0pt}

% ---------- ROS Service ---------- %
\begin{tabularx}{\linewidth}[ht]{@{}LR@{}}
    \multicolumn{2}{@{}l@{}}{\headbf{ROS Service}} \\
    \mytoprule
    \cmd{rosservice list}                          & list active services \hlx
    \cmd{rosservice call \vb{SERVICE} \vb{ARGS}}   & call \vb{SERVICE} with \vb{ARGS} \hlx
    \cmd{rosservice find \vb{TYPE}}                & find services of \vb{TYPE} \hlx
    \cmd{rosservice info \vb{SERVICE}}             & print information about \vb{SERVICE} \hlx
    \cmd{rosservice type \vb{SERVICE}}             & print type of \vb{SERVICE} \hlx
    \cmd{rosservice uri \vb{SERVICE}}              & print uri of \vb{SERVICE} \hlx
    \cmd{rossrv show \vb{TYPE}}                    & print structure of \vb{TYPE} \\
    \mybottomrule
\end{tabularx}
\bigskip

% ---------- ROS Environmental Variables ---------- %
\begin{tabularx}{\linewidth}[ht]{@{}LR@{}}
    \multicolumn{2}{@{}l@{}}{\headbf{ROS Environmental Variables}} \\
    \mytoprule
    \code{ROS\_ROOT}            & location of core ROS packages \hlx
    \code{ROS\_MASTER\_URI}     & location of the master \hlx
    \code{ROS\_PACKAGE\_PATH}   & location for more ROS packages \hlx
    \code{ROS\_HOSTNAME}        & network address of a node \hlx
    \code{ROS\_IP}              & IP address of a node \\
    \mybottomrule
\end{tabularx}
\bigskip

% ---------- ROS Bag ---------- %
\begin{tabularx}{\linewidth}[ht]{@{}LR@{}}
    \multicolumn{2}{@{}l@{}}{\headbf{ROS Bag}} \\
    \mytoprule
    \cmd{rosbag record \vb{TOPIC}}      & record \vb{TOPIC} into bag \hlx
    \cmd{rosbag info \vb{BAG}}          & print content summary of \vb{BAG} \hlx
    \cmd{rosbag play \vb{BAG}}          & play back content of \vb{BAG} \hlx
    \cmd{rosbag check \vb{BAG}}         & check play-ability of \vb{BAG} in current system \hlx
    \cmd{rosbag compress \vb{BAG}}      & compress \vb{BAG} using BZ2 \hlx
    \cmd{rosbag decompress \vb{BAG}}    & decompress \vb{BAG} using BZ2 \\
    \mybottomrule
\end{tabularx}
When simulating in ROS, remember \cb{\cmd{set use\_sim\_time true}} and to append \cb{\code{{-}{-}clock}}.
\bigskip

% ---------- ROS Visualization Tools ---------- %
\begin{tabularx}{\linewidth}[ht]{@{}LR@{}}
    \multicolumn{2}{@{}l@{}}{\headbf{ROS Visualization Tools}} \\
    \mytoprule
    \cmd{rviz}              & 3D visualization of data and models \hlx
    \cmd{gzclient}          & Gazebo GUI \hlx
    \cmd{rqt}               & powerful GUI tool \\
    \cmd{rqt\_plot}         & simple and lightweight plotting \\
    \cmd{rqt\_bag}          & visualize content of a bag \\
    \cmd{rqt\_image\_view}  & visualize camera images \\
    \cmd{rqt\_graph}        & visualize computation graph \\
    \cmd{rqt\_tf\_tree}     & visualize TF frame tree \\
    \mybottomrule
\end{tabularx}
\bigskip

% ---------- ROS Packge Structure ---------- %
\begin{tabularx}{\linewidth}[ht]{@{}LR@{}}
    \multicolumn{2}{@{}l@{}}{\headbf{ROS Packge Structure}} \\
    \mytoprule
    \code{package.xml}       & manifest, dependencies and plugins \hlx
    \code{CMakeLists.txt}    & description of compilation procedure \hlx
    \code{src/}              & C and C++ source codes \hlx
    \code{build/}            & generated makefiles and support files \hlx
    \code{devel/}            & compiled binaries, libraries, headers \hlx
    \code{include/}          & C and C++ header files \hlx
    \code{scripts/}          & Python and bash scripts \hlx
    \code{config/}           & YMAL configuration files \hlx
    \code{cfg/}              & dynamic reconfigure scripts \hlx
    \code{launch/}           & launch files \\
    \mybottomrule
\end{tabularx}

\vspace{0pt}
\columnbreak
\vspace{0pt}

% ---------- ROS TF2 ---------- %
\begin{tabularx}{\linewidth}[ht]{@{}L@{}}
    { \begin{tabularx}{\linewidth}[ht]{@{}L@{}}
            \multicolumn{1}{@{}l@{}}{\headbf{ROS TF2}} \\
            \mytoprule
            \cmd{rosrun tf tf\_echo \vb{FRAME1} \vb{FRAME2}} \\
            print coordinate frame relationship between \vb{FRAME1} and \vb{FRAME2} \\
            Eg. \$ rosrun tf tf\_echo /map /odom \hlx
    \end{tabularx} } \\
    { \begin{tabularx}{\linewidth}[ht]{@{}LR@{}}
            \cmd{rosrun tf view frames}            & visualize coordinate transform tree \\
            \mybottomrule
    \end{tabularx} } \\
\end{tabularx}
tf2 is a power package to deal with coordinate \textbf{t}rans\textbf{f}orm. It maintains the relationship between coordinate frames in a tree structure buffered in time, and lets the user transform points, vectors, etc between any two coordinate frames at any desired point in time.
\bigskip

% ---------- ROS Launch File ---------- %
\begin{tabularx}{\linewidth}[ht]{@{}L@{}}
    { \begin{tabularx}{\linewidth}[ht]{@{}L@{}}
            \multicolumn{1}{@{}l@{}}{\headbf{ROS Launch File}} \\
            \mytoprule
            \xmltag{node name=\vb{NODE} pkg=\vb{PACKAGE} type=\vb{EXE} [args=\vb{ARGS}]} \\
            launch \vb{NODE} using the \vb{EXE} from \vb{PACKAGE} with command line arguments \vb{ARGS} \\
            Eg. \xmltag{node name="rosbag\_record" pkg="rosbag" type="record" args="-a" output="screen"} \hlx
    \end{tabularx} } \\
    { \begin{tabularx}{\linewidth}[ht]{@{}LR@{}}
            \xmltag{include file=\vb{LAUNCHFILE}}            & import \vb{LAUNCHFILE} into the current one \\
    \end{tabularx} } \\
    { \begin{tabularx}{\linewidth}[ht]{@{}L@{}}
            Eg. \xmltag{include file="\$(smb\_local\_planner)/launch/local\_planner.launch"}  \hlx
    \end{tabularx} } \\
    { \begin{tabularx}{\linewidth}[ht]{@{}L@{}R@{}}
            \xmltag{arg name=\vb{ARG}}                       & declare the existence of \vb{ARG} \\
            \xmltag{arg name=\vb{ARG} value=\vb{VAL}}        & declare \vb{ARG} with constant value \vb{VAL} \\
            \xmltag{arg name=\vb{ARG} default=\vb{VAL}}      & declare \vb{ARG} with default value \vb{VAL} \\
    \end{tabularx} } \\
    { \begin{tabularx}{\linewidth}[ht]{@{}L@{}}
        Eg. <arg name="rviz" value="true"/> \hlx
    \end{tabularx} } \\
    { \begin{tabularx}{\linewidth}[ht]{@{}L@{}R@{}}
        \xmltag{param name=\vb{PARAM} value=\vb{VAL}}        & set \vb{PARAM} to \vb{VAL} \\
    \end{tabularx} } \\
    { \begin{tabularx}{\linewidth}[ht]{@{}L@{}}
            {Eg. }\xmltag{param name="frequency" value="300"} \hlx
    \end{tabularx} } \\
    { \begin{tabularx}{\linewidth}[ht]{@{}L@{}R@{}}
            \xmltag{remap from=\vb{OLD} to=\vb{NEW}}        & remap name \vb{OLD} to name \vb{NEW} \\
    \end{tabularx} } \\
    { \begin{tabularx}{\linewidth}[ht]{@{}L@{}}
            {Eg. }\xmltag{remap from="/base\_pose\_measured" to="/base\_pose\_measured\_disabled"} \\
            \mybottomrule
    \end{tabularx} } \\
\end{tabularx}
\bigskip

\vfill\null

\end{multicols*}
\end{document}
